\cleardoublepage

\chapter*{Resumen\markboth{Resumen}{Resumen}}

La conducción autónoma se refiere al uso de sistemas informáticos y tecnologías de inteligencia artificial para permitir que un vehículo se desplace de un lugar a otro sin necesidad de la intervención humana. esta tecnología representa una de las revoluciones tecnológicas más grandes y significativas del siglo XXI. Los pequeños avances que se logran día a día en este ámbito nos ponen un poquito más cerca de un futuro que tiempo atrás parecía utópico e inalcanzable, en el que las personas podremos pasarle el testigo de la movilizacion humana a las máquinas y dejarlas encargadas por completo de nuestro transporte a lo largo de ciudades, países y continentes.
\bigskip

La conducción autónoma se encuentra actualmente en una fase intermedia de su desarrollo, con numerosos avances y despliegues parciales que ofrecen un vistazo al futuro de la movilidad. A pesar de que todavía no hemos alcanzado un nivel de autonomía completo en todas las situaciones y entornos, empresas líderes en el sector como Tesla, Waymo y Cruise ya han introducido sistemas avanzados de asistencia al conductor y, en algunos casos, servicios de transporte sin conductor en áreas específicas. Aunque estas implementaciones actuales suelen requerir la supervisión humana y están principalmente limitadas a escenarios y ubicaciones bien definidos.

\bigskip

A pesar del estado actual de esta tecnología la visión de futuro para los vehículos autónomos es ambiciosa. Se espera que las innovaciones en \ac{IA}, sensores y computación contribuyan a superar las limitaciones actuales. La meta es alcanzar un nivel de autonomía en el que los vehículos puedan navegar de forma segura y eficiente en cualquier entorno o condición climática, sin necesidad de intervención humana. Este avance no solo cambiaría la forma en que nos movemos, sino que también tendría implicaciones profundas en la organización de las ciudades, el uso de la tierra y la calidad de vida. Sin embargo, para que esta visión se convierta en una realidad, todavía hay obstáculos considerables que abordar, incluyendo cuestiones regulatorias y dilemas éticos sobre cómo los vehículos deben comportarse en situaciones imprevistas.
\bigskip

En este proyecto se busca aportar un pequeño avance más al mundo de la conducción autónoma, explorando la aplicación del aprendizaje por refuerzo para crear sistemas de seguimiento de carril, de detención ante obstáculos y de adaptación al tráfico, utilizando el simulador CARLA como plataforma experimental y de desarrollo.

\bigskip
\bigskip
\bigskip
\bigskip
\bigskip
\bigskip
\bigskip
\bigskip
\bigskip
\bigskip
\bigskip
\bigskip
\bigskip
\bigskip

\bigskip
\bigskip
\bigskip
\bigskip
\bigskip
\bigskip
\bigskip
\bigskip
\bigskip
\bigskip
\bigskip
\bigskip
\bigskip
\bigskip

\bigskip
\bigskip
\bigskip
\bigskip
\bigskip
\bigskip
\bigskip
\bigskip
\bigskip
\bigskip
\bigskip
\bigskip
\bigskip
\bigskip
\bigskip
\bigskip
\bigskip
\bigskip
\bigskip
\bigskip
\bigskip
\bigskip
\bigskip
\bigskip
Conducción autonoma en CARLA con adaptación al tráfico basado en aprendizaje por refuerzo © 2023 by Juan Camilo Carmona Sánchez is licensed under Attribution-ShareAlike 4.0 International 


